\documentclass[a4paper]{article}
\usepackage[margin=2cm]{geometry}

\usepackage{fontspec}
\usepackage{animate}
\usepackage{graphicx}
\usepackage[catalan]{babel}
\usepackage{hyperref}
\usepackage{float}
\usepackage{array}
\usepackage[table, dvipsnames]{xcolor}
\usepackage{subcaption}
\usepackage{amsmath}
\usepackage{amsfonts}
\usepackage{amssymb}

\hypersetup{
	colorlinks = true,
	linkcolor = black,
	urlcolor = blue,
}

\setlength{\parindent}{0pt}
\setlength{\parskip}{0.2cm}

\renewcommand{\arraystretch}{1.5}
\setlength\arrayrulewidth{1.1pt}
\rowcolors{2}{gray!25}{white}
\arrayrulecolor{black!60}

\begin{document}
\begin{titlepage}
	\centering
	\vspace{1cm}
	\includegraphics[width=0.25\textwidth]{images/etseib}
	\par\vspace{1cm}
	\textsc{ \LARGE Escola Tècnica Superior d'Enginyeria \\[1em] 
		Industrial de Barcelona}
	\par\vspace{2cm}
	\textbf{\Huge Tècniques Estadístiques per la Qualitat}
	\par\vspace{2cm}
	{\LARGE \textsc{Treball pràctic} \\[1em] Millora de la qualitat del rentat d'apatita}
	\vfill
	\begin{flushright}
		\large
		\texttt{Grup 21} \par
		Marc Asenjo i Ponce de León \par
		Joan Marcè i Igual \par
		Iñigo Moreno i Caireta \par
		Esteve Tarragó Sanchís \par
	\end{flushright}
\end{titlepage}

\tableofcontents
\pagebreak

\section{Introducció}
En aquest treball s'han usat les tècniques de disseny d'experiments per poder veure el funcionament d'un sistema. Per tal de poder tenir una motivació més alta, s'ha contactat amb l'empresa \emph{\href{http://www.protein.es/}{Proteïn S.A.}} de manera que el cas presentat fos un real. 

\subsection{Tema escollit}
L'activitat principal de \emph{Proteïn S.A.} és fabricar  \emph{\href{https://es.wikipedia.org/wiki/Col\%C3\%A1geno_hidrolizado}{co\l.lagen hidrolitzat}} a partir de l'ós de porc que és usat a la indústria de l'alimentació per afegir els nutrients necessaris a la carn. 

A part, un subproducte que es genera és el \emph{\href{https://es.wikipedia.org/wiki/Fosfato_tric\%C3\%A1lcico}{Fosfat Tricàlcic}} ($Ca_3 (PO_4)_2$) o també conegut com \emph{apatita} i un dels usos més comuns és per les gallines que ponen ous perquè no es quedin sense calci. 

\begin{figure}[H]
	\centering
	\includegraphics[width=.8\textwidth]{images/protein/final-compressed}
	\caption{Resultat final de l'apatita}
	\label{fig:Final}
\end{figure} 	

El problema principal de l'apatita és que els compradors la volen amb un \% de fòsfor en massa igual o superior al 15\%. Si aquesta condició no es pot garantir els clients reclamaran indemnitzacions o aniran a comprar a la competència. Així doncs, aquest serà l'objecte principal de l'estudi ja que l'empresa vol assegurar-se que la qualitat de la seva apatita és bona. 

A part també hi ha co\l.lagen que s'escapa per la línia de l'apatita a l'hora de separar els dos productes pel que també pot ser d'interès minimitzar la quantitat que es perd perquè no s'ha pogut separar prou bé.

\subsection{Objectius}
Donat que els clients poden reclamar si la qualitat d'apatita no és suficient i que hi ha part de co\l.lagen que es perd per la línia de l'apatita ens interessa complir els següents objectius:

\begin{itemize}
	\item Maximitzar el \% en massa de fosfor a l'apatita
	\item Minimitzar el \% en massa de co\l.lagen adsorbit a l'apatita
\end{itemize}

\section{Disseny}

\subsection{Sistema a estudiar}
Per poder saber quins factors s'han de controlar cal entendre primer com funciona la part del procés amb el que s'obté l'apatita, veure \autoref{fig:esquema}. Inicialment el producte arriba del procés anterior i conté co\l.lagen i apatita en suspensió en aigua (línia marró al P\&D). Així doncs, s'ha de separar l'apatita de l'aigua i dissoldre el co\l.lagen adherit a l'apatita en suspensió, el problema és que precipita molt fàcilment i s'ha de forçar que es mantingui en suspensió. Per tant, hi ha una sèrie d'elements que asseguren que això sigui així; al dipòsit on va a parar (C-03) hi ha un parell de barrejadors i una bomba de recirculació (P-C32). 

\begin{figure}[H]
	\centering
	\includegraphics[width=0.8\textwidth]{images/esquema}
	\caption{Esquema del sistema a estudiar}
	\label{fig:esquema}
\end{figure}

Per tal de separar el co\l.lagen i l'apatita s'usa un \emph{\href{https://youtu.be/uvWcLZWM_JY}{tricànter}} (veure \autoref{fig:tricanter}), tot i que en aquest procés només funciona com a \emph{decànter}. Aquesta màquina el que fa és, mitjançant força centrífuga separar els sòlids dels líquids dels greixos, no hi ha greix en aquest cas. Cal tenir en compte que aquesta separació no és perfecta i que part del co\l.lagen s'escapa junt amb l'apatita que s'obté.

Per fer la separació la màquina consta d'un cilindre exterior que gira a alta velocitat i un vis sens fi interior que gira a una mica més ràpid que l'exterior, de manera que el vis sens fi va desplaçant lentament el sòlid cap al final del tricànter, situat a l'esquerra de la \autoref{fig:tricanter}, i el líquid cap al principi. 

Aquesta diferència de velocitats entre els cilindres interiors i exteriors influeix en la velocitat a la que s'elimina el sòlid de dins de la màquina i també en la quantitat d'aigua que contindrà.

\begin{figure}[H]
	\centering
	\includegraphics[width=0.8\textwidth]{images/tricanter}
	\caption{Esquema d'un tricànter}
	\label{fig:tricanter}
\end{figure}

\begin{figure}[H]
	\centering
	\animategraphics[draft, autoplay, loop, poster =234, width = 0.5\textwidth]{15}{images/tricanter-movie/tricanter-movie-}{0}{384}
	\caption{Animació explicativa de com funciona un tricanter}
\end{figure}

\subsection{Obtenció i mesura d'una mostra}
Per obtenir una mostra simplement s'agafa de la sortida del tricànter una mica de producte i després es porta al laboratori on es mesurarà.

El procés de mesura és una mica més complicat i consta d'una sèrie d'etapes:

\begin{enumerate}
	\item Es pesa la mostra que conté, entre d'altres, aigua, greix, co\l.lagen i fosfat.
	\item Es posa en un forn a 100ºC per tal d'evaporar tota l'aigua 
	\item Es torna a pesar i es pot veure la quantitat d'aigua que s'ha eliminat
	\item Es posa de nou en un forn però ara a 400ºC per tal d'eliminar mitjançant piròlisi tota la matèria orgànica.
	\item Es torna a pesar i així es pot veure per composició estequiomètrica la quantitat d'apatita en \% en massa que hi havia a la mostra presa.
\end{enumerate} 

Un cop s'ha trobat la massa d'apatita es pot saber el \% en massa de fòsfor de la mostra per estequiometria. El valor teòric és $\% fosfor = 0,2 \cdot \% apatita$ però, donat que el cristall d'apatita porta aigua el real és $\% fosfor = 0,185 \cdot \% apatita$. A part, cal considerar que no tota l'aigua s'arriba a evaporar i que hi ha un petit \% de greix que porta apatita, que normalment es considera constant però que pot ser una font potencial d'errors, aquest valor mitjà sol ser de 0,1459. Per tant:

$$
\frac{\% \text{fòsfor (P)}}{0,185} = \% apatita = 
\frac{W_{ap}}{W_{ap} + W_{H_2O + \text{greix}} + W_{\text{co\l.làgen}}} =
\frac{1}{1 + 0,1459 + \% \text{co\l.làgen en apatita}}
$$

\subsection{Factors i respostes}

Un cop s'ha vist el sistema a estudiar es poden veure els factors que s'hauran de controlar per tal de realitzar l'estudi:

\begin{itemize}
	\item Diferència de velocitats entre els cilindres interior i exterior
	\item Cabal d'entrada al tricànter T-05 (L/h)
	\item Velocitat de la bomba P-C32 (en Hz)
	\item Volum al dipòsit C-03 (en kg)
\end{itemize}

\begin{table}[H]
	\centering
	\begin{tabular}{ l | l | r | r }
		\rowcolor{gray!60}
		Variable Minitab & Factor & Nivell baix & Nivell alt \\ \hline
		A & Diferencial & 8 & 12 \\
		B & Cabal H20 (L/h)& 1800 & 2400 \\
		C & Velocitat P-C32 (Hz) & 50 & 75 \\
		D & Volum C-03 (kg) & 1600 & 2000 \\
	\end{tabular}
	\caption{Nivells escollits pels factors}
\end{table}

La variable de resposta que analitzarem serà el percentatge d'apatita en massa, que intentarem maximitzar variant els factors.
%TODO: WHY?


\subsection{Nombre d'experiments i restriccions externes}
Per tal de realitzar els experiments s'ha decidit amb l'empresa de prendre una mostra cada 12 hores. Aquest interval s'ha decidit així per assegurar-se que el sistema està en equilibri a l'hora de prendre una mostra i que no hi ha altres factors que alteren el resultat, ja que si el sistema es trobés en un estat transitori el resultat es podria veure afectar i això no interessa.

Amb aquest interval de 12 hores entre mostres es calcula que es podran realitzar tots els experiments d'un disseny factorial complet $2^4$ començant a prendre les mostres el 24 de novembre de 2016 i havent acabat als voltants del 16 de desembre de 2016. Tenint en compte que a la fabrica hi poden haver averies o altres incidents que impedeixin realitzar l'experiment.

A part, els valors dels factors no es poden alterar tant fàcilment ja que això implicaria fer canvis molt bruscs en el sistema que s'està controlant de manera que el que es farà és alterar l'ordre de els experiments perquè s'hagi de canviar el mínim nombre de nivells dels factors d'un experiment a un altre.

\subsection{Fonament teòric}
Aquest sistema també ha estat estudiat matemàticament per part de l'empresa pel que el fet de tenir l'estudi ajudarà a comprovar empíricament els valors obtinguts. Cal tenir en compte que en aquest estudi teòric només s'han tingut en compte el valor de dos factors, el cabal d'aigua d'entrada al dipòsit i el volum en estat estacionari al dipòsit C-03, tenint en compte que tots els altres paràmetres són constants, veure \autoref{fig:sistema}.

\begin{figure}[H]
		\centering
		\includegraphics[width=0.5\textwidth]{images/graphs/sistema}
		\caption{Esquema simplificat del sistema teòric}
		\label{fig:sistema}
\end{figure}

\begin{figure}[H]
	\begin{subfigure}{0.46\textwidth}
		\centering
		\includegraphics[width=\textwidth]{images/graphs/apatita}
		\caption{\% en massa d'apatita en funció del cabal d'aigua d'entrada i del volum al C-03}
	\end{subfigure}
	\hfill
	\begin{subfigure}{0.46\textwidth}
		\centering
		\includegraphics[width=\textwidth]{images/graphs/colagen-apatita}
		\caption{\% de co\l.lagen adsorbit a l'apatita en funció del cabal d'aigua d'entrada i del volum del C-03}
	\end{subfigure}
	\caption{Resultats del sistema teòric}
\end{figure}

Així doncs donat que interessa maximitzar el \% en massa d'apatita i minimitzar el \% en massa de co\l.lagen adsorbit a l'apatita teòricament els valors ideals haurien de ser $Q_{H_2O} = 2500 L/h$ i $V_{C-03} = 2000 kg$.

\section{Resultats de l'experimentació}
A partir de les mostres agafades, Proteïn ha proporcionat les dades de \autoref{tab:resultats}. Es pot observar que l'ordre dels experiments no és cronològic ja que com s'ha comentat anteriorment entre experiment i experiment s'havien de fer el mínim de canvis. També, algunes de les proves amb el diferencial (factor A) a 8 es van haver de repetir ja que el parell que s'estava proporcionant a la màquina no era prou alt i això feia que no s'arribés a assolir un diferencial de 8.

\begin{figure}[H]
	\centering
	\includegraphics[width=.5\textwidth]{images/protein/mostra-compressed}
	\caption{Una de les mostres agafades}
	\label{fig:Mostra}
\end{figure} 	

\begin{table}[H]
	\centering
	\begin{tabular}{ l | l | l | l | l | l | l | l | l | l }
		\rowcolor{gray!50}
		FECHA & Hora & A & B & C & D & \%RS & \%C & \%C sms & \%P sms \\ \hline
		02/12/2016 & 16:00 & 8 & 1800 & 50 & 1600 & 55.64 & 47.18 & 82.51 & 15.26 \\ 
		16/12/2016 & 16:00 & 8 & 1800 & 50 & 2000 & 56.23 & 47.88 & 82.85 & 15.33 \\ 
		02/12/2016 & 00:30 & 8 & 1800 & 75 & 1600 & 56.04 & 47.13 & 81.83 & 15.14 \\ 
		16/12/2016 & 07:30 & 8 & 1800 & 75 & 2000 & 56.32 & 47.29 & 81.70 & 15.11 \\ 
		26/11/2016 & 06:00 & 8 & 2400 & 50 & 1600 & 54.82 & 46.35 & 82.27 & 15.21 \\ 
		13/12/2016 & 17:35 & 8 & 2400 & 50 & 2000 & 56.50 & 47.02 & 80.97 & 14.98 \\ 
		02/12/2016 & 09:00 & 8 & 2400 & 75 & 1600 & 54.73 & 46.48 & 82.63 & 15.29 \\ 
		07/12/2016 & 15:00 & 8 & 2400 & 75 & 2000 & 56.57 & 46.81 & 80.51 & 14.89 \\ 
		13/12/2016 & 20:15 & 12 & 1800 & 50 & 1600 & 56.41 & 47.78 & 82.41 & 15.25 \\ 
		13/12/2016 & 01:30 & 12 & 1800 & 50 & 2000 & 54.48 & 45.46 & 81.19 & 15.02 \\ 
		25/11/2016 & 20:30 & 12 & 1800 & 75 & 1600 & 54.82 & 46.35 & 82.27 & 15.22 \\ 
		14/12/2016 & 19:30 & 12 & 1800 & 75 & 2000 & 55.96 & 46.55 & 80.94 & 14.97 \\ 
		25/11/2016 & 16:25 & 12 & 2400 & 50 & 1600 & 55.11 & 45.27 & 79.93 & 14.79 \\ 
		01/12/2016 & 17:45 & 12 & 2400 & 50 & 2000 & 55.95 & 48.06 & 83.58 & 15.47 \\ 
		15/12/2016 & 19:30 & 12 & 2400 & 75 & 1600 & 54.03 & 42.35 & 76.27 & 14.11 \\ 
		06/12/2016 & 15:00 & 12 & 2400 & 75 & 2000 & 55.29 & 45.86 & 80.70 & 14.93 \\ 
	\end{tabular}
	\caption{Dades experimentals, ordenades pels nivells dels factors}
	\label{tab:resultats}
\end{table}

\section{Anàlisi}
Duent a terme l’anàlisi factorial sobre les dades es decideix prendre com a resposta la concentració (\% en massa) d’apatita de la mostra. Amb aquest estudi s’obtenen els diagrames Normal i de Pareto que es poden observar en les següents figures. Al costat s’han afegit els diagrames obtinguts seleccionant com a resposta la concentració (\% en massa) de co\l.làgen. Aquests serveixen per comprovar que aquestes dues respostes estan relacionades fermament, degut que si una concentració augmenta l’altre disminueix per conseqüència.
\begin{figure}[H]
	\begin{subfigure}{.5\textwidth}
		\centering
		\includegraphics[width=.9\linewidth]{images/NormalEffects15P}
		\caption{}
		\label{fig:NormalEffects15P}
	\end{subfigure}%
	\begin{subfigure}{.5\textwidth}
		\centering
		\includegraphics[width=.9\linewidth]{images/NormalEffects15C}
		\caption{}
		\label{fig:NormalEffects15C}
	\end{subfigure}
	\caption{Normal Plots obtinguts amb l’anàlisi factorial tenint com a resposta \%P sms (a) i \%C sms (b)}
	\label{fig:NormalEffects15}
\end{figure}

\begin{figure}[H]
	\begin{subfigure}{.5\textwidth}
		\centering
		\includegraphics[width=.9\linewidth]{images/Pareto15P}
		\caption{}
		\label{fig:Pareto15P}
	\end{subfigure}%
	\begin{subfigure}{.5\textwidth}
		\centering
		\includegraphics[width=.9\linewidth]{images/Pareto15C}
		\caption{}
		\label{fig:Pareto15C}
	\end{subfigure}
	\caption{Diagrames de Pareto obtinguts amb l’anàlisi factorial tenint com a resposta \%P sms (a) i \%C sms (b)}
	\label{fig:Pareto15}
\end{figure}

S’observa en un primer instant que tots els efectes dels factors seleccionats tenen valors bastant petits. Per poder considerar algun dels efectes com a significatiu l’alfa ha de ser de 0,15, quan el valor per defecte és de 0,05. Això es pot explicar pel fel fet que els resultats obtinguts pels valors de les respostes no són del tot exactes, sinó que comporten un error significatiu. Això és degut a les suposicions que s’han pres en la mesura de la mostra, com per exemple que tota l’aigua s’ha evaporat o que el \% de greix en l’apatita és constant. També es duu a terme una conversió per estimar la quantitat de fòsfor en l’apatita, per tal de tenir en compte que els cristalls d’aquesta contenen aigua, i dita conversió és aproximada. Per totes aquestes raons la variància i l’error en els valors de resposta augmenta, cosa que fa que els efectes dels factors en comparació disminueixin.

D’altra banda es pot apreciar que el factor que amb diferència té més efecte i que és considerat com a significatiu és l’ABD, que correspon a la combinació de Diferencial, Cabal i Volum de C-O3. També es veu que tant el Diferencial com el Cabal tenen efecte negatiu en la resposta. Això és degut a que essent reduïts tots dos fan que la mescla passi més estona en el tricanter, i que per tant la separació es faci millor. El que no té tant sentit aparent és que aquests dos factors tinguin efecte només en conjunt amb el Volum de C-O3, però es decideix acceptar el resultat com a correcte de totes maneres. Els factors individualment semblen tenir efectes que, encara que negligibles, ens ajudaran a prendre decisions a continuació.

Gràcies a que el anàlisi que s'ha fet es un $2^4$ i no un $2^{3-1}$ s'ha pogut trobar que el efecte mes significatiu es l'ABD, ja que amb el $2^{3-1}$ l'haguéssim confós amb el C. 

Es comprova que, efectivament, els resultats obtinguts fent l’estudi amb \%C i \%P sms com a resposta són directament dependents, casi idèntics, amb una diferència d’escala deguda als valors mitjans de les concentracions d’aquests elements.
 
Per determinar ara els valors desitjats d’aquestes variables es duu a terme l’Interaction Plot que veiem a continuació, i s’analitzen els resultats reportats per Minitab.

\begin{figure}[H]
	\centering
	\includegraphics[width=.5\textwidth]{images/Coefs}
	\caption{Efectes i coeficients en la fórmula final de \%P sms}
	\label{fig:Coefs}
\end{figure} 	

\begin{figure}[H]
	\centering
	\includegraphics[width=.5\textwidth]{images/InteractionP}
	\caption{Efectes i coeficients en la fórmula final de \%P sms}
	\label{fig:InteractionP}
\end{figure} 

Veient les interaccions entre les tres variables i els seus efectes (que tot i què segons els normal plots de la \autoref{fig:NormalEffects15} no son significatius, no són negligibles) s’ha decidit donar valors baixos a les variables A i B i valor alt a la D. A més, a la \autoref{fig:InteractionP} es pot veure que si fiquem un valor alt a D(C-O3) es redueix l'efecte de les variables A i B, deixant el resultat sempre per sobre del 15\%. Això es bó ja que segurament al escollir un valor alt de D tindrem menys variabilitat en els resultats 	complirem mes sovint les restriccions del 15\%. Escollint aquests valors s’aconsegueix que ABD tingui un efecte positiu i que individualment cada una de les variables també ajudin a augmentar la resposta.

Pel que respecta al factor C, s’ha decidit baix ja que el seu efecte era negatiu en la resposta i representa una reducció de cost per l’empresa. Recorda que la variable C representa la velocitat de la bomba de recirculació del dipòsit i s’estalviarà electricitat reduint la velocitat. 

\section{Conclusions}
Hem acabat concloent que el factor amb un efecte més significatiu en la concentració de fosfat en la mostra final és la interacció entre el diferencial, el cabal i el volum en C-O3. Encara que aquest no és el resultat que esperàvem des d’un principi, s’ha acabat veient que és raonable. En canvi, la bomba de recirculació no ha mostrat tenir un efecte massa significatiu a diferencia del que haviem previst.

Per extreure’n les conclusions definitives caldrà fer un seguiment de la resposta amb els valors dels factors que hem especificat per comprovar que el nivell de fosfat supera els requeriments amb els nous factors, i comparar els valors amb els obtinguts amb els valors actuals per comprovar que hi ha hagut una millora.

Aquest treball ens ha permès aplicar els coneixements teòrics adquirits durant l’assignatura a un problema real derivat d’una empresa.
Tot i la bona disposició de l’empresa alhora de facilitar-nos les dades i explicar-nos el funcionament de la fàbrica, el món laboral presenta diversos reptes que hem hagut d’afrontar, com ara un numero d’experiments fixat i una bona planificació dels mateixos.

Estem satisfets de la col·laboració amb l’empresa ja que creiem que ambdós n’hem sortit beneficiats. L’empresa ens ha facilitat les dades i ha realitzat els experiments i nosaltres hem fet una anàlisi per què millorin la qualitat d’un dels seus productes.


\end{document}